\documentclass[11pt,a4paper,sans]{moderncv} 

\moderncvstyle{classic} 
\moderncvcolor{green} 
\usepackage[scale=0.75]{geometry}
\setlength{\hintscolumnwidth}{2.5cm}
\usepackage[utf8]{inputenc}

% WHOAMI 
\name{Maël}{GRAMAIN}
\title{Étudiant en réseaux et télécommunications}
\address{97170 Petit-Bourg}{}{Guadeloupe}
\phone[mobile]{+590 690 24 55 69} 
\email{maelgramain@gmail.com} 
\homepage{www.enpls.org} 
\social[linkedin]{maelgramain}
% \photo[64pt][0pt]{picture.jpeg}

% DETAILS
\begin{document}
\makecvtitle

\section{Formation}
\cventry{Prévu 2022-}{BUT Réseaux et Télécommunications en Alternance}{IUT de Saint-Malo}{}{Formation en Alternance}{} 

\cventry{2022}{Concours GEIPI Polytech}{}{}{\textbf{Admis}}{}

\cventry{2019-2022}{Baccalauréat Mathématiques et Physique Chimie}{Lycée Général et Technologique des Droits de l'Homme}{Petit-Bourg, Guadeloupe}{\textit{}}{
\begin{itemize}
\item 3ème spécialité : Numérique Sciences de l’Informatique
\item LVA : Anglais, LVB : Espagnol, LVC : Allemand
\item Classe euro-caribéenne anglais/histoire 
\end{itemize}}


\section{Compétences}
\cvitem{Développement}{HTML/CSS, GoLang, Python3, BASH}
\cvitem{Système}{GNU/Linux, Docker, Proxmox, MariaDB/MySQL, PostgreSQL, NGINX, Traefik}
\cvitem{Réseau}{IPv4/6, BGP, Cisco IOS, MikroTik RouterOS, FRRouting}
\cvitem{Français}{Langue maternelle}
\cvitem{Anglais}{Lu, écrit, parlé (Niveau B2)}
\cvitem{Espagnol}{Lu, écrit, parlé (Niveau B1)}
\cvitem{Allemand}{Notions (Niveau A2)}
\cvitem{Personnelles}{Travail en équipe, Autonomie, Autodidacte}



\section{Expériences professionnelles}
\cventry{Février 2019}{Technicien en Télécommunications}{Alliance Delta Line Technologies}{Jarry}{Stage de 1 semaine}{Déploiement de clients ADSL et FTTH en sous traitance Orange. Interventions en centrale téléphonique.}

\clearpage

\section{Projets Associatifs}
\cventry{Mars 2021 - Aujourd’hui}{Association MilkyWan}{Fournisseur d'Accès à Internet associatif}{}{}{
\begin{itemize}
\item Développement Web (\textbf{HTML5}, \textbf{CSS3}, JavaScript)
\item Développement d’outils internes en \textbf{Go} et \textbf{Python} (automatisation à l'aide d'\textbf{API REST}, \textbf{supervision})
\item \textbf{Intégration et livraison continue} avec GitLab
\item Maintenance de systèmes \textbf{GNU/Linux virtualisés} et d’applications internes
\item Gestion IPAM et DCIM (Netbox)
\end{itemize}}

\cventry{Février 2017 - Aujourd’hui}{GnousEU}{Communauté centrée sur le logiciel libre}{}{}{
\begin{itemize}
\item Maintenance de systèmes \textbf{GNU/Linux}, d’applications Python, Ruby et Golang. Gestion de base de données \textbf{MariaDB} (MySQL), \textbf{PostgreSQL} et \textbf{Redis}.
\item Gestion DNS (\textbf{PowerDNS} et \textbf{Knot}) et Web (\textbf{NGINX}, \textbf{Traefik})
\item Supervision : \textbf{SNMPv3}, \textbf{Observium}, Stack \textbf{Grafana} (Telegraf et Prometheus)
\item Automatisation : Python, BASH, Ansible
\item Déploiement d’applications containerisées (containerd, docker, docker-compose)
\item Déploiement du réseau \textbf{AS213253} avec les technologies \textbf{BGP} et \textbf{OSPFv3} sous \textbf{FRRouting}
\end{itemize}}

\section{Centres d'intérêt}
\cvitem{Loisir}{Cinéma d’auteur et d’animation, Musique folk, Photographie}
\cvitem{Sports}{Kayak, Natation, Randonnée}

\end{document}
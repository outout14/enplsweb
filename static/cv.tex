\documentclass[10pt,a4paper,sans]{moderncv} 

\moderncvstyle{classic} 
\moderncvcolor{green} 
\usepackage[scale=0.75]{geometry}
\setlength{\hintscolumnwidth}{2.5cm}
\usepackage[utf8]{inputenc}

% WHOAMI 
\name{Maël}{GRAMAIN}
\title{Administrateur Systèmes et Réseaux}
\address{35700 Rennes}{Bretagne, France}{}
\phone[mobile]{+33 6 38 91 33 73} 
\email{maelgramain@gmail.com} 
\homepage{www.enpls.org} 
\social[linkedin]{maelgramain}
% \photo[64pt][0pt]{picture.jpeg}

% DETAILS
\begin{document}
\makecvtitle

\section{Formations}
\cventry{2022-2024}{BUT Réseaux et Télécommunications}{IUT de Saint-Malo}{}{}{
\begin{itemize}
    \item Mathématiques du signal, électronique et télécommunications
    \item Réseaux opérateur et d'entreprise, xDSL, GPON, transmission hertzienne
    \item Programmation orientée objet, administration système et virtualisation
    \item Communication, anglais technique
\end{itemize}
}
\cventry{2019-2022}{Baccalauréat Mathématiques et Physique-Chimie}{Lycée des Droits de l'Homme}{Guadeloupe}{}{
\begin{itemize}
\item Admis mention Très Bien
\end{itemize}}

\section{Expérience professionnelle}
\cventry{Septembre 2022 - Maintenant}{Ingénieur réseau}{OVHCloud}{Cesson-Sévigné}{Alternance}{
Au sein de l'équipe Backbone (AS16276) : 
\begin{itemize}
    \item Développement d'outils internes en microservice
    \item Automatisation de processus
    \item Recherche et qualification d'équipements réseau
    \item Implémentation d'architectures réseau
\end{itemize}}

\section{Compétences}
\cvitem{Développement}{GoLang, Python3, Microservices, VueJS, Apache Airflow, CI/CD}
\cvitem{Système}{GNU/Linux, Kubernetes, Docker, Grafana, Ansible}
\cvitem{Base de données}{SQL (PostgreSQL, MariaDB), NoSQL (MongoDB), InfluxDB, Apache Druid}
\cvitem{Réseau}{IPv4/6, BGP, OSPF, Cisco IOS(-XR), JunOS, FRRouting}
\cvitem{Linguistiques}{Français, Anglais (B2), Espagnol (B1)}
\cvitem{Personnelles}{Travail en équipe, Autonomie, Autodidacte}

\section{Projets associatifs}
\cventry{Octobre 2022}{CTF Flagmalo}{Évennement de cybersécurité}{IUT de Saint-Malo}{}{Responsable de l'infrastructure système et réseau. Mise en place de la plateforme de virtualisation (Proxmox), architecture réseau (Mikrotik, WiFi Cisco) et supervision (Grafana, InfluxDB).}

\cventry{Mars 2021 - Octobre 2022}{Association MilkyWan}{Fournisseur d'Accès à Internet associatif}{}{}{Administration des systèmes GNU/Linux. Gestion des peering avec les ISP et CSP. Développement Web et logiciels internes. Gestion IPAM et DCIM. }

\cventry{Février 2017 - Aujourd’hui}{GnousEU}{Communauté centrée sur le logiciel libre}{}{}{Administration des systèmes GNU/Linux et logiciels associés. Gestion du réseau AS213253.}

\section{Centres d'intérêt}
\cvitem{Loisir}{Cinéma d’auteur et d’animation, Musique folk, Photographie}
\cvitem{Sports}{Kayak, Natation, Randonnée}

\end{document}
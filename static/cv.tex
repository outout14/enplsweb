\documentclass[10pt,a4paper,sans]{moderncv} 

\moderncvstyle{classic} 
\moderncvcolor{green} 
\usepackage[scale=0.80]{geometry}
\setlength{\hintscolumnwidth}{2.5cm}
\usepackage[utf8]{inputenc}

% WHOAMI 
\name{Maël}{GRAMAIN}
\title{Aspirant ingénieur réseau et télécom}
\address{35700 Rennes}{Bretagne, France}{}
\phone[mobile]{+33 6 38 91 33 73} 
\email{maelgramain@ieee.org} 
\homepage{www.enpls.org} 
\social[linkedin]{maelgramain}
\extrainfo{Permis B}

% DETAILS
\begin{document}
\makecvtitle

\section{Formations}
\cventry{2024-2027}{Diplôme d'Ingénieur}{IMT Atlantique}{Spécialité informatique, réseaux, télécommunications}{en apprentissage}{
\begin{itemize}
    \item Programme théorique en mathématiques du signal, électronique et physique des télécommunications
    \item Programmation et architecture logicielle bas et haut niveau, réseaux IP et VoIP
    \item Gestion de projet, management d'équipe
    \item Communication, anglais technique
\end{itemize}
}
\cventry{2022-2024}{BUT Réseaux et Télécommunications}{IUT de Saint-Malo}{}{en apprentissage}{
\begin{itemize}
    \item Réseaux opérateur et d'entreprise, xDSL, GPON, transmissions hertziennes
    \item Programmation orientée objet, administration système
    \item Parcours DevCloud : Containerisation, virtualisation et microservices 
\end{itemize}
}
\cventry{2019-2022}{Baccalauréat Mathématiques et Physique-Chimie}{Lycée des Droits de l'Homme}{Guadeloupe}{}{}

\section{Expérience professionnelle}
\cventry{Septembre 2022 - Aujourd’hui}{Apprenti ingénieur réseau}{OVHCloud}{Cesson-Sévigné}{Alternance}{
Au sein de l'équipe Backbone (AS16276) : 
\begin{itemize}
    \item Développement d'outils internes en microservices
    \item Automatisation de processus
    \item Recherche et qualification d'équipements réseau
    \item Implémentation d'architectures réseau
\end{itemize}}

\section{Compétences}
\cvitem{Développement}{GoLang, Python3, Microservices, VueJS, Apache Airflow, CI/CD}
\cvitem{Systèmes}{GNU/Linux, Kubernetes, Docker, Grafana, Ansible}
\cvitem{Bases de données}{SQL (PostgreSQL, MariaDB), NoSQL (MongoDB), InfluxDB, Apache Druid}
\cvitem{Réseaux}{IPv4/6, BGP, OSPF, Cisco IOS(-XR), JunOS, FRRouting}
\cvitem{Linguistiques}{Français, anglais (C1), espagnol (B1)}
\cvitem{Personnelles}{Travail en équipe, autonomie, autodidacte}

\section{Projets associatifs}
\cventry{Octobre 2022}{CTF Flagmalo}{Évènement de cybersécurité}{IUT de Saint-Malo}{}{Responsable de l'infrastructure système et réseau. Mise en place de la plateforme de virtualisation, architecture réseau et supervision.}
\cventry{Mars 2021 - Octobre 2022}{Association MilkyWan}{Fournisseur d'Accès à Internet associatif}{}{}{Administration des systèmes GNU/Linux. Gestion des interconnexions. Programmation Web et de logiciels internes. Gestion IPAM et DCIM. }

\cventry{Février 2017 - Aujourd’hui}{GnousEU}{Communauté centrée sur le logiciel libre}{}{}{Administration des systèmes GNU/Linux et logiciels associés. Gestion du réseau AS213253.}

\section{Centres d'intérêt}
\cvitem{Loisir}{Cinéma d’auteur et d’animation, photographie, mécanique}
\cvitem{Sports}{Kayak, natation, randonnée}

\end{document}